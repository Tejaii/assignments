\documentclass[journal]{IEEEtran}
\usepackage[a5paper, margin=10mm, onecolumn]{geometry}
\usepackage{amsmath, amssymb, amsfonts}
\usepackage{graphicx}
\usepackage{xcolor}
\usepackage{enumitem}
\usepackage{multicol}
\usepackage{mathtools}

 % Define the custom commands
\newcommand{\brak}[1]{\left( #1 \right)}
\newcommand{\sbrak}[1]{\left[ #1 \right]}
\newcommand{\abs}[1]{\left| #1 \right|}
\newcommand{\lt}{<}
\newcommand{\gt}{>}

\begin{document}
\bibliographystyle{IEEEtran}
\title{ASSIGNMENT 13}
\author{EE24BTECH11034 - K Teja Vardhan}
{\let\newpage\relax\maketitle}

\section{JEE PYQ 2024 January 30, shift 2}
\begin{enumerate}

    \item Let $f\brak{x}=\begin{cases}
    x-1, & x\text{ is even} \\
    2x, & x\text{ is odd}
    \end{cases}$, $x\in\mathbb{N}$. If for some $a\in\mathbb{N}$, $f\brak{f\brak{f\brak{a}}}=21$, then $\lim_{x\to a}\sbrak{\frac{\abs{x}^{3}}{a}-\sbrak{\frac{x}{a}}}$ is equal to:
    
        \begin{multicols}{4}
        \begin{enumerate}
        \item $121$
        \item $144$
        \item $169$
        \item $225$
        \end{enumerate}
        \end{multicols}

    \item Let the system of equations $x+2y+3z=5$, $2x+3y+z=9$, $4x+3y+\lambda z=\mu$ have infinite number of solutions. Then $\lambda+2\mu$ is equal to:
        
        \begin{multicols}{4}
        \begin{enumerate}
        \item $28$
        \item $17$
        \item $22$
        \item $15$
        \end{enumerate}
        \end{multicols}
        
    \item Consider $10$ observations $x_1, x_2, \ldots, x_{10}$ such that $\sum_{i=1}^{10}\brak{x_i-\alpha}=2$ and $\sum_{i=1}^{10}\brak{x_i-\beta}^{2}=40$, where $\alpha, \beta$ are positive integers. Let the mean and the variance of the observations be $\frac{6}{5}$ and $\frac{84}{25}$, respectively. The $\frac{\beta}{\alpha}$ is equal to:
        
        \begin{multicols}{4}
        \begin{enumerate}
        \item $2$
        \item $\frac{3}{2}$
        \item $\frac{5}{2}$
        \item $1$
        \end{enumerate}
        \end{multicols}

    \item Let Ajay will not appear in JEE exam with probability $p=\frac{2}{7}$, while both Ajay and Vijay will appear in the exam with probability $q=\frac{1}{5}$. Then the probability, that Ajay will appear in the exam and Vijay will not appear is:

        \begin{multicols}{4}
        \begin{enumerate}
        \item $\frac{9}{35}$
        \item $\frac{18}{35}$
        \item $\frac{24}{35}$
        \item $\frac{3}{35}$
        \end{enumerate}
        \end{multicols}
        
    \item Let the locus of the mid points of the chords of the circle $x^{2}+\brak{y-1}^{2}=1$ drawn from the origin intersect the line $x+y=1$ at $P$ and $Q$. Then, the length of $PQ$ is:

        \begin{multicols}{4}
        \begin{enumerate}
        \item $\frac{1}{\sqrt{2}}$
        \item $\sqrt{2}$
        \item $\frac{1}{2}$
        \item $1$
        \end{enumerate}
        \end{multicols}
        
    \item If three successive terms of a G.P. with common ratio $r\brak{r>1}$ are the lengths of the sides of a triangle and $\sbrak{r}$ denotes the greatest integer less than or equal to $r$, then $3\sbrak{r}+\sbrak{-r}$ is equal to:

        \begin{multicols}{4}
        \begin{enumerate}
        \item $1$
        \item $7$
        \item $8$
        \item $9$
        \end{enumerate}
        \end{multicols}

    \item Let $A=I_{2}-2MM^{T}$, where $M$ is a real matrix of order $2\times1$ such that the relation $M^{T}M=I_{1}$ holds. If $\lambda$ is a real number such that the relation $AX=\lambda X$ holds for some non-zero real matrix $X$ of order $2\times1$, then the sum of squares of all possible values of $\lambda$ is equal to:

        \begin{multicols}{4}
        \begin{enumerate}
        \item $1$
        \item $2$
        \item $3$
        \item $4$
        \end{enumerate}
        \end{multicols}
                
    \item Let $f\brak{\brak{0,\infty}}\rightarrow\mathbb{R}$ and $F\brak{x}=\int_{0}^{x}tf\brak{t}dt$. If $F\brak{x^{2}}=x^{4}+x^{5}$, then $\sum_{r=1}^{12}f\brak{r^{2}}$ is equal to:

        \begin{multicols}{4}
        \begin{enumerate}
        \item $219$
        \item $144$
        \item $156$
        \item $168$
        \end{enumerate}
        \end{multicols}   

    \item If $y=\frac{\brak{\sqrt{x}+1}\brak{x^{2}-\sqrt{x}}}{x\sqrt{x}+x+\sqrt{x}}+\frac{1}{15}\brak{3\cos^{2}x-5}\cos^{3}x$, then $96y\brak{\frac{\pi}{6}}$ is equal to:

        \begin{multicols}{4}
        \begin{enumerate}
        \item $105$
        \item $13$
        \item $15$
        \item $17$
        \end{enumerate}
        \end{multicols}

    \item Let $\vec{a}=\hat{i}+\hat{j}+\hat{k}$, $\vec{b}=-\hat{i}-8\hat{j}+2\hat{k}$, and $\vec{c}=4\hat{i}+c_{2}\hat{j}+c_{3}\hat{k}$ be three vectors such that $\vec{b}\times\vec{a}=\vec{c}\times\vec{a}$. If the angle between the vector $\vec{c}$ and the vector $3\hat{i}+4\hat{j}+\hat{k}$ is $0$, then the greatest integer less than or equal to $\tan^{2}\theta$ is:

        \begin{multicols}{4}
        \begin{enumerate}
        \item $38$
        \item $2$
        \item $3$
        \item $4$
        \end{enumerate}
        \end{multicols}

    \item The lines $L_{1}, L_{2}, \ldots, L_{20}$ are distinct. For $n=1, 2, 3, \ldots, 10$ all the lines $L_{2n-1}$ are parallel to each other and all the lines $L_{2n}$ pass through a given point $P$. The maximum number of points of intersection of pairs of lines from the set $\sbrak{L_{1},L_{2},...,L_{20}}$ is equal to:

        \begin{multicols}{4}
        \begin{enumerate}
        \item $101$
        \item $191$
        \item $192$
        \item $193$
        \end{enumerate}
        \end{multicols}

    \item Three points $O\brak{0,0}, P\brak{a,a^{2}}, Q\brak{-b,b^{2}}, a\gt0, b\gt0$ are on the parabola $y=x^{2}$. Let $S_{1}$ be the area of the region bounded by the line $PQ$ and the parabola, and $S_{2}$ be the area of the triangle $OPQ$. If the minimum value of $\frac{S_{1}}{S_{2}}$ is $\frac{m}{n}$, $\gcd\brak{m,n}=1$, then $m+n$ is equal to:

        \begin{multicols}{4}
        \begin{enumerate}
        \item $7$
        \item $8$
        \item $9$
        \item $10$
        \end{enumerate}
        \end{multicols}

    \item The sum of squares of all possible values of $k$, for which area of the region bounded by the parabolas $2y^{2}=kx$ and $ky^{2}=2\brak{y-x}$ is maximum, is equal to:

        \begin{multicols}{4}
        \begin{enumerate}
        \item $10$
        \item $8$
        \item $14$
        \item $16$
        \end{enumerate}
        \end{multicols}

    \item If $\frac{dx}{dy}=\frac{1+x-y^{2}}{y}$, $x\brak{1}=1$, then $5x\brak{2}$ is equal to:

        \begin{multicols}{4}
        \begin{enumerate}
        \item $11$
        \item $5$
        \item $15$
        \item $17$
        \end{enumerate}
        \end{multicols}

    \item Let $ABC$ be an isosceles triangle in which $A$ is at $\brak{-1,0}$, $\angle A=\frac{2\pi}{3}$, $AB=AC$ and $B$ is on the positive $x$-axis. If $BC=4\sqrt{3}$ and the line $BC$ intersects the line $y=x+3$ at $\brak{\alpha,\beta}$, then $\frac{\beta^{4}}{\alpha^{2}}$ is:

        \begin{multicols}{4}
        \begin{enumerate}
        \item $36$
        \item $32$
        \item $64$
        \item $128$
        \end{enumerate}
        \end{multicols}       
\end{enumerate}
\end{document}
