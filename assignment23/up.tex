\documentclass{article}
\usepackage{tikz, tkz-euclide}
\usepackage{circuitikz}
\usepackage{siunitx}
\usepackage[a5paper, margin=10mm, onecolumn]{geometry}
\usepackage{amsmath, amssymb, amsfonts}
\usepackage{graphicx}
\usepackage{xcolor}
\usepackage{mathrsfs}
\usepackage{pgfplots}
\usepackage{enumitem}
\usepackage{multicol}
\usepackage{mathtools}
\DeclareMathOperator*{\infimum}{inf}

% Define the custom commands
\newcommand{\brak}[1]{\left( #1 \right)}
\newcommand{\sbrak}[1]{\left[ #1 \right]}
\newcommand{\abs}[1]{\left| #1 \right|}
\newcommand{\lt}{<}
\newcommand{\gt}{>}
\renewcommand{\prime}{\text{\textquotesingle}}

\begin{document}
\bibliographystyle{IEEEtran}
\title{2018-ME-27-39}
\author{EE24BTECH11034 - K Teja Vardhan}
{\let\newpage\relax\maketitle}

\begin{enumerate}

\item An infinite solenoid carries a time-varying current $I\brak{t} = At^2$, with $A \neq 0$. The axis of the solenoid is along the $z$ direction. $\hat{r}$ and $\hat{\theta}$ are the usual radial and polar directions in cylindrical polar coordinates. $\vec{B} = B_r \hat{r} + B_{\theta} \hat{\theta} + B_z \hat{z}$ is the magnetic field at a point outside the solenoid.

Which one of the following statements is true?  

\begin{enumerate}
    \item  $B_r = 0, B_{\theta} = 0, B_z = 0$
    \item  $B_r \neq 0, B_{\theta} \neq 0, B_z = 0$
    \item  $B_r \neq 0, B_{\theta} \neq 0, B_z \neq 0$
    \item  $B_r = 0, B_{\theta} = 0, B_z \neq 0$
\end{enumerate}

\item A uniform volume charge density is placed inside a conductor with resistivity $10^{-2}~\Omega\text{m}$. The charge density becomes $\frac{1}{2.718}$ of its original value after time in femtoseconds  $\brak{\varepsilon_0 = 8.854 \times 10^{-12}~\frac{F}{m}}$.

\item Water freezes at $0^\circ$C at atmospheric pressure $1.01 \times 10^5$ Pa. The densities of water and ice at this temperature and pressure are $1000~\frac{kg}{m}^3$ and $934~\frac{kg}{m}^3$ respectively. The latent heat of fusion is $3.34 \times 10^5~\frac{J}{kg}$. The pressure required for depressing the melting temperature of ice by $10^\circ$C is in .

\item The minimum number of NAND gates required to construct an OR gate is:

\begin{enumerate}
    \item $2$
    \item $4$
    \item $5$
    \item $3$
\end{enumerate}

\item Consider a $2$-dimensional electron gas with a density of $10^{19}~\text{m}^{-2}$. The Fermi energy of the system is eV . 

$\brak{m_e = 9.31 \times 10^{-31}~\text{kg}, h = 6.626 \times 10^{-34}~\text{Js}, e = 1.602 \times 10^{-19}~\text{C}}$

\item The total energy of an inert-gas crystal is given by $E\brak{R} = \dfrac{0.5}{R^{12}} - \dfrac{1}{R^6}$ , where $R$ is the inter-atomic spacing in Angstroms. The equilibrium separation between the atoms is Angstroms .

\item Consider $N$ non-interacting, distinguishable particles in a two-level system at temperature $T$. The energies of the levels are $0$ and  
 $\epsilon$, where $\epsilon \gt 0$. In the high temperature limit $\brak{k_B T \gg \epsilon}$, what is the population of particles in the level with energy $\epsilon$?

\begin{enumerate}
    \item $\dfrac{N}{2}$
    \item $N$
    \item $\dfrac{N}{4}$
    \item $\dfrac{3N}{4}$
\end{enumerate}

\item Consider a one-dimensional potential well of width $3$ nm. Using the uncertainty principle $\brak{\Delta x \Delta p \geq \dfrac{\hbar}{2}}$, an estimate of the minimum depth of the well such that it has at least one bound state for an electron is :

\begin{enumerate}
    \item $1 \mu eV$
    \item $1$ meV
    \item $1$ eV
    \item $1$ MeV
\end{enumerate}

\item A free electron of energy $1$ eV is incident upon a one-dimensional finite potential step of height $0.75$ eV. The probability of its reflection from the barrier is .

 \item Consider a metal with free electron density of $6 \times 10^{22}~\text{cm}^{-3}$. The lowest frequency electromagnetic radiation to which this metal is transparent is $1.38 \times 10^{16}~\text{Hz}$. If this metal had a free electron density of $1.8 \times 10^{23}~\text{cm}^{-3}$ instead, the lowest frequency electromagnetic radiation to which it would be transparent is  $\times 10^{16}~\text{Hz}$ .

\item An object travels along the $x$-direction with velocity $\frac{c}{2}$ in a frame $O$. An observer in a frame $O \prime$ sees the same object travelling with velocity $\frac{c}{4}$. The relative velocity of $O'$ with respect to $O$ in units of $c$ is  .

 \item The integral $\int_{0}^{\infty} x^2 e^{-x^2} dx$ is equal to  .

 \item The imaginary part of an analytic complex function is $v\brak{x,y} = 2xy + 3y$. The real part of the function is zero at the origin. The value of the real part of the function at $1 + i$ is .

\end{enumerate}
\end{document}
