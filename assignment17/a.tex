\documentclass[journal]{IEEEtran}
\usepackage[a5paper, margin=10mm, onecolumn]{geometry}
\usepackage{amsmath, amssymb, amsfonts}
\usepackage{graphicx}
\usepackage{xcolor}
\usepackage{mathrsfs}
\usepackage{enumitem}
\usepackage{multicol}
\usepackage{mathtools}

% Define the custom commands
\newcommand{\brak}[1]{\left( #1 \right)}
\newcommand{\sbrak}[1]{\left[ #1 \right]}
\newcommand{\abs}[1]{\left| #1 \right|}
\newcommand{\lt}{<}
\newcommand{\gt}{>}

\begin{document}
\bibliographystyle{IEEEtran}
\title{2014-AE-53-55}
\author{EE24BTECH11034 - K Teja Vardhan}
{\let\newpage\relax\maketitle}


\begin{enumerate}



\item Which of the following design parameters influence the maximum rate-of-climb for a jet-propelled airplane?

\begin{itemize}
    \item Wing loading
    \item Maximum thrust-to-weight ratio
    \item Zero-lift drag coefficient
    \item Maximum lift-to-drag ratio
\end{itemize}

\begin{enumerate}
    \item $P$ and $Q$ alone
    \item $P$, $Q$, $R$, and $S$
    \item $P$, $Q$, and $S$ alone
    \item $Q$, $R$, and $S$ alone
\end{enumerate}

\item Consider the following four statements regarding aircraft longitudinal stability:

\begin{itemize}
    \item $C_{M,cg}$ at zero lift must be positive
    \item $\frac{\partial C_{M,cg}}{\partial \alpha_a}$ must be negative :$\alpha_a$ is absolute angle of attack
    \item $C_{M,cg}$ at zero lift must be negative
    \item Slope of $C_L$ versus $\alpha_a$ must be negative
\end{itemize}

Which of the following combination is the necessary criterion for stick fixed longitudinal balance and static stability?


\begin{enumerate}
    \item $Q$ and $R$ only
    \item $Q$, $R$, and $S$ only
    \item $P$ and $Q$ only
    \item $Q$ and $S$ only
\end{enumerate}

\item Data for a light, single-engine, propeller-driven aircraft in steady level flight at sea level is as follows: velocity $V_{\infty} = 40 \, \frac{m}{s}$, weight $W = 13000 \, \text{N}$, lift coefficient $C_L = 0.65$, drag coefficient $C_D = 0.025 + 0.04 C_L^2$, and power available $P_{av} = 100,000 \, \frac{J}{s}$. The rate of climb possible for this aircraft under the given conditions is


\begin{enumerate}
    \item $7.20$
    \item $5.11$
    \item $6.32$
    \item $4.23$
\end{enumerate}






\end{enumerate}
\end{document}
