\documentclass[journal]{IEEEtran}
\usepackage[a5paper, margin=10mm, onecolumn]{geometry}
\usepackage{amsmath, amssymb, amsfonts}
\usepackage{graphicx}
\usepackage{xcolor}
\usepackage{mathrsfs}
\usepackage{enumitem}
\usepackage{multicol}
\usepackage{mathtools}

% Define the custom commands
\newcommand{\brak}[1]{\left( #1 \right)}
\newcommand{\sbrak}[1]{\left[ #1 \right]}
\newcommand{\abs}[1]{\left| #1 \right|}
\newcommand{\lt}{<}
\newcommand{\gt}{>}

\begin{document}
\bibliographystyle{IEEEtran}
\title{2007-MA-18-34}
\author{EE24BTECH11034 - K Teja Vardhan}
{\let\newpage\relax\maketitle}

\section{JEE PYQ 2024 January 30, shift 2}
\begin{enumerate}

  \item If $F\brak{s} = \tan^{-1}\brak{s} + k$ is the Laplace transform of some function $f\brak{t}$, $t \geq 0$, then $k$ is:
    
    \begin{multicols}{4}
    \begin{enumerate}
        \item $-\pi$
        \item $-\frac{\pi}{2}$
        \item $0$
        \item $\frac{\pi}{2}$
    \end{enumerate}
    \end{multicols}

  \item Suppose $y_p\brak{x} = x \cos\brak{2x}$ is a particular solution of $y\prime\prime + \alpha y = -4 \sin\brak{2x}$. Then the constant $\alpha$ equals:
    
    \begin{multicols}{4}
    \begin{enumerate}
        \item $-4$
        \item $-2$
        \item $2$
        \item $4$
    \end{enumerate}
    \end{multicols}

  \item Let $S = \sbrak{\brak{0,1,1}, \brak{1,0,1}, \brak{-1,2,1}} \subseteq \mathbb{R}^3$. Suppose $\mathbb{R}^3$ is endowed with the standard inner product $\langle \cdot, \cdot \rangle$. Define $M = \sbrak{x \in \mathbb{R}^3 : \langle x, y \rangle = 0 \text{ for all } y \in S}$. Then the dimension of $M$ equals:

    \begin{multicols}{4}
    \begin{enumerate}
        \item $0$
        \item $1$
        \item $2$
        \item $3$
    \end{enumerate}
    \end{multicols}

  \item Let $X$ be an uncountable set and let $\mathscr{T} = \sbrak{ U \subseteq X : U = \emptyset \text{ or } U^c \text{ is finite} }$. Then the topological space $\brak{X, \mathscr{T}}$:

    \begin{enumerate}
        \item is separable
        \item is Hausdorff
        \item has a countable basis
        \item has a countable basis at each point
    \end{enumerate}

  \item Suppose $\brak{X, \mathscr{T}}$ is a topological space. Let $\sbrak{S_n}_{n \geq 1}$ be a sequence of subsets of $X$. Then:

     \begin{enumerate}
        \item $\brak{S_1 \cup S_2}^c = S_1^c \cup S_2^c$
        \item $\brak{ \bigcup_n S_n }^c = \bigcup_n S_n^c$
        \item $\bigcup_n S_n = \bigcup_n \overline{S_n}$
        \item $\overline{S_1 \cup S_2} = \overline{S_1} \cup \overline{S_2}$
    \end{enumerate}
  
  \item Suppose $\brak{X, d}$ is a metric space. Consider the metric $\rho$ on $X$ defined by:
    $
    \rho\brak{x, y} = \min\sbrak{ \frac{1}{2}, d\brak{x, y} }, \quad x, y \in X.
    $
    Suppose $\mathscr{T}_1$ and $\mathscr{T}_2$ are topologies on $X$ defined by $d$ and $\rho$, respectively. Then:

    \begin{enumerate}
        \item $\mathscr{T}_1$ is a proper subset of $\mathscr{T}_2$
        \item $\mathscr{T}_2$ is a proper subset of $\mathscr{T}_1$
        \item Neither $\mathscr{T}_1 \subseteq \mathscr{T}_2$ nor $\mathscr{T}_2 \subseteq \mathscr{T}_1$
        \item $\mathscr{T}_1 = \mathscr{T}_2$
    \end{enumerate}

  \item A basis of $V = \sbrak{\brak{x, y, z, w} \in \mathbb{R}^4 : x + y - z = 0, \, y + z + w = 0, \, 2x + y - 3z - w = 0 }$ is:

    \begin{enumerate}
        \item $\sbrak{\brak{1, 1, -1, 0}, \brak{0, 1, 1, 1}, \brak{2, 1, -3, 1}}$
        \item $\sbrak{\brak{1, -1, 0, 1}}$
        \item $\sbrak{\brak{1, 0, 1, -1}}$
        \item $\sbrak{\brak{1, -1, 0, 1}, \brak{1, 0, 1, -1}}$
    \end{enumerate}
    

   \item Consider $\mathbb{R}^3$ with the standard inner product. Let
   $S=\sbrak{\brak{1,1,1}, \brak{2,-1,2}, \brak{1,-2,1}}.$

    For a subset $W$ of $\mathbb{R}^3$, let $L\brak{W}$ denote the linear span of $W$ in $\mathbb{R}^3$. Then an orthonormal set $T$ with $L\brak{S}=L\brak{T}$ is 
  
    \begin{enumerate}
      \item $\sbrak{\frac{1}{\sqrt{3}}\brak{1,1,1}, \frac{1}{\sqrt{6}}\brak{1,-2,1}}$
      \item $\sbrak{\brak{1,0,0}, \brak{0,1,0}, \brak{0,0,1}}$
      \item $\sbrak{\frac{1}{\sqrt{3}}\brak{1,1,1}, \frac{1}{\sqrt{2}}\brak{1,-1,0}}$
      \item $\sbrak{\frac{1}{\sqrt{3}}\brak{1,1,1}, \frac{1}{\sqrt{2}}\brak{0,1,-1}}$
    \end{enumerate}

  \item Let $A$ be a $3\times3$ matrix. Suppose that the eigenvalues of $A$ are $-1,0,1$ with respective
eigenvectors $\brak{1,-1,0}^T$, $\brak{1,1,-2}^T$, and $\brak{1,1,1}^T$. Then $6A$ equals:

    \begin{enumerate}
      \item $\begin{bmatrix}
        -1 & 5 & 2 \\
        5 & -1 & 2 \\
        2 & 2 & 2
      \end{bmatrix}$
      \item $\begin{bmatrix}
        0 & 0 & 0 \\
        0 & -1 & 0 \\
        0 & 0 & 0
      \end{bmatrix}$
      \item $\begin{bmatrix}
        -2 & 5 & 2 \\
        5 & -1 & 2 \\
        2 & 2 & 2
      \end{bmatrix}$
      \item None of the above
    \end{enumerate}

  \item A matrix is called row stochastic if the sum of each of its row entries is $1$. Suppose $P$ is a row stochastic matrix of size $4 \times 4$. Then for any $\mathbf{x} \in \mathbb{R}^4$ and for any $n \geq 0$,
  $
  \mathbf{1}^T P^n \mathbf{x} =
  $
    \begin{enumerate}
      \item $1$
      \item $0$
      \item $\mathbf{1}^T \mathbf{x}$
      \item $\mathbf{x}$
    \end{enumerate}

  \item Consider the matrix $A = \begin{bmatrix}
        1 & 1 \\
        1 & 1
      \end{bmatrix}$. Let $B = A^n$ for $n \geq 1$. Then $B$ equals:
    \begin{enumerate}
      \item $2^{n-1} A$
      \item $nA$
      \item $2^{n-2} A$
      \item $A^2$
    \end{enumerate}

\item Let
  $u\brak{x,y}=f\brak{xe^{x}} + g\brak{y^2\cos\brak{y}},$
  where $f$ and $g$ are infinitely differentiable functions. Then the partial differential
  equation of minimum order satisfied by $u$ is  

  \begin{enumerate}
    \item $u_{xx} + x u_{xy} = u_{x}$
    \item $u_{yy} + x u_{xx} = x u_{x}$
    \item $u_{yy} - x u_{xy} = u_{x}$
    \item $u_{yy} - x u_{xx} = x u_{x}$
  \end{enumerate}

\item Let $C$ be the boundary of the triangle formed by the points $\brak{1,0,0}$, $\brak{0,1,0}$, $\brak{0,0,1}$. Then the value of the line integral
  $
  \int_C -2y\, dx + \brak{3x - 4y^2}\, dy + \brak{z^2 + 3y}\, dz
  $
  is

  \begin{enumerate}
    \item $0$
    \item $1$
    \item $2$
    \item $4$
  \end{enumerate}

\item Let $X$ be a complete metric space and let $E \subseteq X$. Consider the following statements:
  \begin{enumerate}
    \item $E$ is compact,
    \item $E$ is closed and bounded,
    \item $E$ is closed and totally bounded,
    \item Every sequence in $E$ has a subsequence converging in $E$.
  \end{enumerate}
  Which one of the above statements does NOT imply all the other statements?

  \begin{multicols}{4}
  \begin{enumerate}
    \item $S_1$
    \item $S_2$
    \item $S_3$
    \item $S_4$
  \end{enumerate}
  \end{multicols}

\item Consider the series
  $\sum_{n=1}^\infty \frac{1}{n^{3/2}}\sin\brak{nx}.$
  Then the series

  \begin{enumerate}
    \item converges uniformly on $\mathbb{R}$
    \item converges pointwise but NOT uniformly on $\mathbb{R}$
    \item converges in $L^1$ norm to an integrable function on $\sbrak{0,2\pi}$ but does NOT
      converge uniformly on $\mathbb{R}$
    \item does NOT converge pointwise
  \end{enumerate}

\item Let $f\brak{z}$ be an analytic function. Then the value of
  $\int_{0}^{2\pi} f\brak{e^{it}} \cos\brak{t} dt$
  equals

  \begin{enumerate}
    \item $0$
    \item $2\pi f\brak{0}$
    \item $2\pi f\prime\brak{0}$
    \item $\pi f\prime\brak{0}$
  \end{enumerate}

\item Let $G_1$ and $G_2$ be the images of the disc $\sbrak{z \in \mathbb{C} : \abs{z+1} \lt 1}$ under the transformations
  $
  w = \frac{\brak{1-i}z + 2}{\brak{1+i}z + 2}
  $
  and
  $
  w = \frac{\brak{1+i}z + 2}{\brak{1-i}z + 2}
  $
  respectively. Then  

  \begin{enumerate}
    \item $G_1 = \sbrak{w \in \mathbb{C} : \operatorname{Im}\brak{w} \lt 0}$ and $G_2 = \sbrak{w \in \mathbb{C} : \operatorname{Im}\brak{w} \gt 0}$
    \item $G_1 = \sbrak{w \in \mathbb{C} : \operatorname{Im}\brak{w} \gt 0}$ and $G_2 = \sbrak{w \in \mathbb{C} : \operatorname{Im}\brak{w} \lt 0}$
    \item $G_1 = \sbrak{w \in \mathbb{C} : \abs{w} \gt 2}$ and $G_2 = \sbrak{w \in \mathbb{C} : \abs{w} \lt 2}$
    \item $G_1 = \sbrak{w \in \mathbb{C} : \abs{w} \lt 2}$ and $G_2 = \sbrak{w \in \mathbb{C} : \abs{w} \gt 2}$
  \end{enumerate}


\end{enumerate}
\end{document}
