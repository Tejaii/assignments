\documentclass[journal]{IEEEtran}
\usepackage[a5paper, margin=10mm, onecolumn]{geometry}
\usepackage{amsmath, amssymb, amsfonts}
\usepackage{graphicx}
\usepackage{xcolor}
\usepackage{enumitem}
\usepackage{multicol}
\usepackage{mathtools}

 % Define the custom commands
\newcommand{\brak}[1]{\left( #1 \right)}
\newcommand{\sbrak}[1]{\left[ #1 \right]}
\newcommand{\abs}[1]{\left| #1 \right|}
\newcommand{\lt}{<}
\newcommand{\gt}{>}


\begin{document}
\bibliographystyle{IEEEtran}
\title{ASSIGNMENT 12}
\author{EE24BTECH11034 - K Teja Vardhan}
{\let\newpage\relax\maketitle}

\section{JEE PYQ 2024 January 30, shift 2}
\begin{enumerate}

    \item Bag $A$ contains $7$ white balls and $3$ red balls. Bag $B$ contains $3$ white balls and $2$ red balls. A ball is chosen randomly and found to be red. Then, find the probability that it is taken from bag $A$:

        \begin{multicols}{4}
        \begin{enumerate}
        \item $\frac{7}{20}$
        \item $\frac{1}{2}$
        \item $\frac{3}{7}$
        \item $\frac{1}{5}$
        \end{enumerate}
        \end{multicols}

    \item Given $\abs{\vec{b}}=2, \abs{\vec{b}\times\vec{a}}=2$

    Then $\abs{\vec{b}\times\vec{a}-\vec{b}}^2$ is
    
        \begin{multicols}{4}
        \begin{enumerate}
        \item $0$
        \item $8$
        \item $1$
        \item $10$
        \end{enumerate}
        \end{multicols}

    \item If $f\brak{x}=\ln\brak{\frac{2x+3}{4x^2-x-3}}+\cos^{-1}\brak{\frac{2x+1}{x+2}}$. If domain of $f\brak{x}$ is $\sbrak{\alpha,\beta}$, then $5\alpha-4\beta$ is

        \begin{multicols}{4}
        \begin{enumerate}
        \item $-2$
        \item $3$
        \item $-4$
        \item $1$
        \end{enumerate}
        \end{multicols}

    \item If $f\brak{x}=\frac{x}{\brak{1+x^4}^{1/4}}$ and $g\brak{x}=f\brak{f\brak{f\brak{f\brak{x}}}}$, then $\int_{0}^{\sqrt{2}-\sqrt{5}}x^2g\brak{x}dx$ is equal to:

        \begin{multicols}{4}
        \begin{enumerate}
        \item $\frac{11}{6}$
        \item $\frac{13}{6}$
        \item $\frac{2}{5}$
        \item $\frac{17}{6}$
        \end{enumerate}
        \end{multicols}
        
    \item If first term of a GP is $a$ and third term is $b$ and in second GP first term is $a$ and fifth term is $b$ and eleventh term of first GP common to which term of second GP:

        \begin{multicols}{4}
        \begin{enumerate}
        \item $24$
        \item $25$
        \item $21$
        \item $18$
        \end{enumerate}
        \end{multicols}

    \item $z^{1985}+z^{100}+1=0$ and $z^3+2z^2+2z+1=0$. Then number of common roots of equation is:

        \begin{multicols}{4}
        \begin{enumerate}
        \item $1$
        \item $2$
        \item $3$
        \item $4$
        \end{enumerate}
        \end{multicols}

    \item If $x^2-y^2+2hxy+2gx+2fy+c=0$ is the locus of points such that it is equidistant from the lines $x+2y-8=0$ and $2x+y+7=0$, then the value of $h+g+f+c$ is:

        \begin{multicols}{4}
        \begin{enumerate}
        \item $15$
        \item $-15$
        \item $20$
        \item $-20$
        \end{enumerate}
        \end{multicols}
        
    \item Let $A=\begin{bmatrix}x&0&0\\0&y&0\\0&0&z\end{bmatrix}$ and $\frac{x}{\sin\theta}=\frac{y}{\sin\left(\theta+\frac{2\pi}{3}\right)}=\frac{z}{\sin\left(\theta+\frac{4\pi}{3}\right)}$. Then

        \begin{enumerate}
        \item Statement 1: $T_{r}\brak{A}=0$ and Statement 2: $T_{r}\brak{\text{adj}\brak{\text{adj}(A}}=0$ are both true.
        \item Statement 1 is true.
        \item Statement 2 is true.
        \item None of these.
        \end{enumerate}
        
    \item If $S_{n}=3+7+11+\cdots$ upto $n$ terms and $40\gt\frac{6}{n\brak{n+1}}\sum_{k=1}^{n}S_{k}\gt45$, then $n$ is:

        \begin{multicols}{4}
        \begin{enumerate}
        \item $9$
        \item $10$
        \item $11$
        \item $12$
        \end{enumerate}
        \end{multicols}

    \item In a paper there are $3$ sections $A$, $B$, and $C$ which has $8$, $6$, and $6$ questions each. A student has to attempt $15$ questions such that they have to attempt at least $4$ questions out of each sections, then number of ways of attempting these questions are:

        \begin{multicols}{4}
        \begin{enumerate}
        \item $11300$
        \item $11376$
        \item $12576$
        \item $13372$
        \end{enumerate}
        \end{multicols}

\end{enumerate}
\end{document}
