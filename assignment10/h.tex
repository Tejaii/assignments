\documentclass[journal]{IEEEtran}
\usepackage[a5paper, margin=10mm, onecolumn]{geometry}
\usepackage{amsmath, amssymb, amsfonts}
\usepackage{graphicx}
\usepackage{xcolor}
\usepackage{enumitem}
\usepackage{multicol}
\usepackage{mathtools}

% Define the custom commands
\newcommand{\brak}[1]{\left( #1 \right)}
\newcommand{\sbrak}[1]{\left[ #1 \right]}
\newcommand{\abs}[1]{\left| #1 \right|}
\newcommand{\lt}{<}
\newcommand{\gt}{>}

\begin{document}
\bibliographystyle{IEEEtran}
\title{ASSIGNMENT 9}
\author{EE24BTECH11034 - K Teja Vardhan}
{\let\newpage\relax\maketitle}

\section{JEE PYQ 2022 July 26, shift 1}
\begin{enumerate}

    \item If $z\ne0$ be a complex number such that $\abs{z-\frac{1}{z}}=2$, then the maximum value of $\abs{z}$ is:
    
        \begin{multicols}{4}
        \begin{enumerate}
        \item $\sqrt{2}$
        \item $1$
        \item $\sqrt{2}-1$
        \item $\sqrt{2}+1$
        \end{enumerate}
        \end{multicols}

    \item Which of the following matrices can NOT be obtained from the matrix $\begin{bmatrix}-1 & 2 \\ 1 & -1\end{bmatrix}$ by a single elementary row operation?

        \begin{multicols}{4}
        \begin{enumerate}
        \item $\begin{bmatrix}0 & 1 \\ 1 & -1\end{bmatrix}$
        \item $\begin{bmatrix}1 & -1 \\ -1 & 2\end{bmatrix}$
        \item $\begin{bmatrix}-1 & 2 \\ -2 & 7\end{bmatrix}$
        \item $\begin{bmatrix}-1 & 2 \\ -1 & 3\end{bmatrix}$
        \end{enumerate}
        \end{multicols}

    \item The system of equations

        $x + y + z = 6$
        
        $2x + 5y + \alpha z = \beta$
        
        $x + 2y + 3z = 14$
        
        has infinitely many solutions. Then $\alpha + \beta$ is equal to:
 
        \begin{multicols}{4}
        \begin{enumerate}
        \item $8$
        \item $36$
        \item $44$
        \item $48$
        \end{enumerate}
        \end{multicols}

    \item Let the function
        $f\brak{x}=\begin{cases}
        \frac{\log_{e}\brak{1+5x}-\log_{e}\brak{1+ax}}{x}, & \text{if } x\neq0 \\
        10, & \text{if } x=0
        \end{cases}$        
        be continuous at $x=0$. The $\alpha$ is equal to:
        
        \begin{multicols}{4}
        \begin{enumerate}
        \item $10$
        \item $-10$
        \item $5$
        \item $-5$
        \end{enumerate}
        \end{multicols}

    \item If $\sbrak{t}$ denotes the greatest integer $\leq t$, then the value of $\int_{0}^{1}\sbrak{2x-13x^{2}-5x+21+1}dx$ is:

        \begin{multicols}{4}
        \begin{enumerate}
        \item $\frac{\sqrt{37}+\sqrt{13}-4}{6}$
        \item $\frac{\sqrt{37}-\sqrt{13}-4}{6}$
        \item $\frac{-\sqrt{37}-\sqrt{13}+4}{6}$
        \item $\frac{-\sqrt{37}+\sqrt{13}+4}{6}$
        \end{enumerate}
        \end{multicols}

    \item Let $\sbrak{a_{n}}_{n=0}^{\infty}$ be a sequence such that $a_{0}=a_{1}=0$ and $a_{n+2}=3a_{n+1}-2a_{n}+1$, $\forall n\geq0$. Then $a_{25}-2a_{23}-2a_{22}+4a_{24}$ is equal to:

        \begin{multicols}{4}
        \begin{enumerate}
        \item $483$
        \item $528$
        \item $575$
        \item $624$
        \end{enumerate}
        \end{multicols}

    \item $\sum_{r=1}^{20}\brak{r^{2}+1}\brak{r!}$ is equal to:

        \begin{multicols}{4}
        \begin{enumerate}
        \item $22!-21!$
        \item $22!-2\brak{21!}$
        \item $21!-2\brak{20!}$
        \item $21!-20!$
        \end{enumerate}
        \end{multicols}

    \item For $I\brak{x}=\int\frac{\sec^{2}x-2022}{\sin^{2022}x}dx$, if $I\brak{\frac{\pi}{4}}=2^{1011}$, then

        \begin{enumerate}
        \item $3^{1010}I\brak{\frac{\pi}{3}}-I\brak{\frac{\pi}{6}}=0$
        \item $3^{1010}I\brak{\frac{\pi}{6}}-I\brak{\frac{\pi}{3}}=0$
        \item $3^{1011}I\brak{\frac{\pi}{3}}-I\brak{\frac{\pi}{6}}=0$
        \item $3^{1011}I\brak{\frac{\pi}{6}}-I\brak{\frac{\pi}{3}}=0$
        \end{enumerate}
      
    \item If the solution curve of the differential equation $\frac{dy}{dx}=\frac{x+y-2}{x-y}$ passes through the point $\brak{2,1}$ and $\brak{k+1,2},k\gt0$, then

        \begin{enumerate}
        \item $2\tan^{-1}\brak{\frac{1}{k}}=\log_{e}\brak{k^{2}+1}$
        \item $\tan^{-1}\brak{\frac{1}{k}}=\log_{e}\brak{k^{2}+1}$
        \item $2\tan^{-1}\brak{\frac{1}{k+1}}=\log_{e}\brak{k^{2}+2k+2}$
        \item $2\tan^{-1}\brak{\frac{1}{k}}=\log_{e}\brak{\frac{k^{2}+1}{k^{2}}}$
        \end{enumerate}

    \item Let $y=y\brak{x}$ be the solution curve of the differential equation $\frac{dy}{dx}+\frac{2x^{2}+11x+13}{x^{3}+6x^{2}+11x+6}y=\frac{x+3}{x+1}$, $x\gt-1$, which passes through the point $\brak{0,1}$. Then $y\brak{1}$ is equal to:

        \begin{multicols}{4}
        \begin{enumerate}
        \item $\frac{1}{2}$
        \item $\frac{3}{2}$
        \item $\frac{5}{2}$
        \item $\frac{7}{2}$
        \end{enumerate}
        \end{multicols}

    \item Let $m_{1}, m_{2}$ be the slopes of two adjacent sides of a square of side $a$ such that $a^{2}+11a+3\brak{m_{1}^{2}+m_{2}^{2}}=220$. If one vertex of the square is $\brak{10\brak{\cos{\alpha}-\sin{\alpha}}, 10\brak{\sin{\alpha}+\cos{\alpha}}}$, where $\alpha\in\brak{0,\frac{\pi}{2}}$, and the equation of one diagonal is $\brak{\cos{\alpha}-\sin{\alpha}}x+\brak{\sin{\alpha}+\cos{\alpha}}y=10$, then $72\brak{\sin^{4}{\alpha}+\cos^{4}{\alpha}}+a^{2}-3a+13$ is equal to:

        \begin{multicols}{4}
        \begin{enumerate}
        \item $119$
        \item $128$
        \item $145$
        \item $155$
        \end{enumerate}
        \end{multicols}

    \item The number of elements in the set $S=\sbrak{x\in\mathbb{R}:2\cos\brak{\frac{x^{2}+x}{6}}=4^{x}+4^{-x}}$ is:

        \begin{multicols}{4}
        \begin{enumerate}
        \item $1$
        \item $3$
        \item $0$
        \item infinite
        \end{enumerate}
        \end{multicols}

    \item Let $A\brak{\alpha,-2}$, $B\brak{\alpha,6}$, and $C\brak{\frac{\alpha}{4},-2}$ be vertices of a $\triangle ABC$. If $\brak{5,\frac{\alpha}{4}}$ is the circumcentre of $\triangle ABC$, then which of the following is NOT correct about $\triangle ABC$:

        \begin{multicols}{4}
        \begin{enumerate}
        \item area is $24$
        \item perimeter is $25$
        \item circumradius is $5$
        \item inradius is $2$
        \end{enumerate}
        \end{multicols}

    \item Let $Q$ be the foot of perpendicular drawn from the point $P\brak{1,2,3}$ to the plane $x+2y+z=14$. If $R$ is a point on the plane such that $\angle PRQ=60^{\circ}$, then the area of $\triangle PQR$ is equal to:

        \begin{multicols}{4}
        \begin{enumerate}
        \item $\frac{\sqrt{3}}{2}$
        \item $\sqrt{3}$
        \item $2\sqrt{3}$
        \item $3$
        \end{enumerate}
        \end{multicols}
                
    \item If $\brak{2,3,9}$, $\brak{5,2,1}$, $\brak{1,\lambda,8}$, and $\brak{\lambda,2,3}$ are coplanar, then the product of all possible values of $\lambda$ is:

        \begin{multicols}{4}
        \begin{enumerate}
        \item $\frac{21}{2}$
        \item $\frac{59}{8}$
        \item $\frac{57}{8}$
        \item $\frac{95}{8}$
        \end{enumerate}
        \end{multicols}
        
\end{enumerate}
\end{document}
