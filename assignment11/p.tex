\documentclass[journal]{IEEEtran}
\usepackage[a5paper, margin=10mm, onecolumn]{geometry}
\usepackage{amsmath, amssymb, amsfonts}
\usepackage{graphicx}
\usepackage{xcolor}
\usepackage{enumitem}
\usepackage{multicol}
\usepackage{mathtools}

 % Define the custom commands
\newcommand{\brak}[1]{\left( #1 \right)}
\newcommand{\sbrak}[1]{\left[ #1 \right]}
\newcommand{\abs}[1]{\left| #1 \right|}
\newcommand{\lt}{<}
\newcommand{\gt}{>}


\begin{document}
\bibliographystyle{IEEEtran}
\title{ASSIGNMENT 11}
\author{EE24BTECH11034 - K Teja Vardhan}
{\let\newpage\relax\maketitle}

\section{JEE PYQ 2023 January 31, shift 1}
\begin{enumerate}

    \item If $\sin^{-1}\brak{\frac{\alpha}{17}} + \cos^{-1}\brak{\frac{4}{5}} - \tan^{-1}\brak{\frac{77}{36}} = 0$, $0 \lt \alpha \lt 13$, then $\sin^{-1}\brak{\sin \alpha} + \cos^{-1}\brak{\cos \alpha}$ is equal to:
        
        \begin{multicols}{4}
        \begin{enumerate}
        \item $\pi$
        \item $16$
        \item $0$
        \item $16 - 5\pi$
        \end{enumerate}
        \end{multicols}
4
    \item Let a circle $C_{1}$ be obtained on rolling the circle $x^{2}+y^{2}-4x-6y+11=0$ upwards $4$ units on the tangent $T$ to it at the point $\brak{3, 2}$. Let $C_{2}$ be the image of $C_{1}$ in $T$. Let $A$ and $B$ be the centers of circles $C_{1}$ and $C_{2}$ respectively, and $M$ and $N$ be respectively the feet of perpendiculars drawn from $A$ and $B$ on the x-axis. Then the area of the trapezium $AMNB$ is:
    
        \begin{multicols}{4}
        \begin{enumerate}
        \item $2\brak{2+\sqrt{2}}$
        \item $4\brak{1+\sqrt{2}}$
        \item $3+2\sqrt{2}$
        \item $2\brak{1+\sqrt{2}}$
        \end{enumerate}
        \end{multicols}
        
    \item S1: $\brak{p\Rightarrow q}\lor\brak{p\land\brak{\neg q}}$ is a tautology.

      S2: $\brak{\brak{\neg p}\Rightarrow\brak{\neg q}}\land\brak{\brak{\neg p}\lor q}$ is a contradiction. Then
    
        \begin{enumerate}
        \item only $\brak{S2}$ is correct
        \item both $\brak{S1}$ and $\brak{S2}$ are correct
        \item both $\brak{S1}$ and $\brak{S2}$ are wrong
        \item only $\brak{S1}$ is correct
        \end{enumerate}
    
    \item The value of $\int_{\frac{\pi}{3}}^{\frac{\pi}{2}}\frac{\brak{2+3\sin{x}}}{\sin{x}\brak{1+\cos{x}}}dx$ is equal to:

        \begin{enumerate}
        \item $\frac{7}{2}-\sqrt{3}-\log_{e}\sqrt{3}$
        \item $-2+3\sqrt{3}+\log_{e}\sqrt{3}$
        \item $\frac{10}{3}-\sqrt{3}+\log_{e}\sqrt{3}$
        \item $\frac{10}{3}-\sqrt{3}-\log_{e}\sqrt{3}$
        \end{enumerate}

    \item A bag contains $6$ balls. Two balls are drawn from it at random and both are found to be black. The probability that the bag contains at least $5$ black balls is:

        \begin{multicols}{4}
        \begin{enumerate}
        \item $\frac{5}{7}$
        \item $\frac{2}{7}$
        \item $\frac{3}{7}$
        \item $\frac{5}{6}$
        \end{enumerate}
        \end{multicols}

    \item Let $5$ digit numbers be constructed using the digits $0, 2, 3, 4, 7, 9$ with repetition allowed, and are arranged in ascending order with serial numbers. Then the serial number of the number $42923$ is:

        \begin{multicols}{4}
        \begin{enumerate}
        \item $2997$
        \item $2998$
        \item $2999$
        \item $2996$
        \end{enumerate}
        \end{multicols}

    \item Let $a_{1}, a_{2},\ldots,a_{n}$ be in A.P. If $a_{5}=2a_{7}$ and $a_{11}=18$, then
    $
    12\left(\frac{1}{\sqrt{a_{10}}+\sqrt{a_{11}}}+\frac{1}{\sqrt{a_{11}}+\sqrt{a_{12}}}+\cdots+\frac{1}{\sqrt{a_{17}}+\sqrt{a_{18}}}\right)
    $
    is equal to:
    
        \begin{multicols}{4}
        \begin{enumerate}
        \item $3$
        \item $4$
        \item $8$
        \item $6$
        \end{enumerate}
        \end{multicols}
   
    \item Let $\theta$ be the angle between the planes $P_{1}=\vec{r}\cdot\brak{\hat{i}+\hat{j}+2\hat{k}}=9$ and $P_{2}=\vec{r}\cdot\brak{2\hat{i}-\hat{j}+\hat{k}}=15$. Let $L$ be the line that meets $P_{2}$ at the point $\brak{4,-2, 5}$ and makes an angle $\theta$ with the normal of $P_{2}$. If $\alpha$ is the angle between $L$ and $P_{2}$, then $\brak{\tan^{2}\theta}\brak{\cot^{2}\alpha}$ is equal to:

        \begin{multicols}{4}
        \begin{enumerate}
        \item $9$
        \item $3$
        \item $\frac{9}{16}$
        \item $\frac{16}{9}$
        \end{enumerate}
        \end{multicols}

    \item Let $\alpha \gt 0$ be the smallest number such that the expansion of $\brak{x^{\frac{2}{3}}+\frac{2}{x^{3}}}^{30}$ has a term $\beta x^{-\alpha}$, $\beta\in\mathbb{N}$. Then $\alpha$ is equal to:

        \begin{multicols}{4}
        \begin{enumerate}
        \item $10$
        \item $2$
        \item $14$
        \item $16$
        \end{enumerate}
        \end{multicols}

    \item Let $\vec{a}$ and $\vec{b}$ be two vectors such that $\abs{\vec{a}}=\sqrt{14}$, $\abs{\vec{b}}=\sqrt{6}$, and $\abs{\vec{a}\times\vec{b}}=\sqrt{48}$. Then $\brak{\vec{a}\cdot\vec{b}}^{2}$ is equal to:

        \begin{multicols}{4}
        \begin{enumerate}
        \item $16$
        \item $25$
        \item $36$
        \item $49$
        \end{enumerate}
        \end{multicols}

    \item Let the line $L:\frac{x-1}{2}=\frac{y+1}{-1}=\frac{z-3}{1}$ intersect the plane $2x+y+3z=16$ at the point $P$. Let the point $Q$ be the foot of perpendicular from the point $R\brak{1,-1,-3}$ on the line $L$. If $\alpha$ is the area of triangle $PQR$, then $\alpha^{2}$ is equal to:

        \begin{multicols}{4}
        \begin{enumerate}
        \item $\frac{16}{3}$
        \item $\frac{25}{3}$
        \item $\frac{36}{3}$
        \item $\frac{49}{3}$
        \end{enumerate}
        \end{multicols}

    \item The remainder on dividing $5^{99}$ by $11$ is:

        \begin{multicols}{4}
        \begin{enumerate}
        \item $1$
        \item $2$
        \item $3$
        \item $4$
        \end{enumerate}
        \end{multicols}
        
    \item If the variance of the frequency distribution
    \begin{tabular}{|c|c|c|c|c|c|c|c|}
    \hline
    $X_{i}$ & 2 & 3 & 4 & 5 & 6 & 7 & 8 \\ \hline
    Frequency $f_{i}$ & 3 & 6 & 16 & $\alpha$ & 9 & 5 & 6 \\ \hline
    \end{tabular}
    is $2.5$, then $\alpha$ is equal to:
    
        \begin{multicols}{4}
        \begin{enumerate}
        \item $7$
        \item $8$
        \item $9$
        \item $10$
        \end{enumerate}
        \end{multicols}

    \item Let for $x\in\mathbb{R}$
     $f\brak{x}=\frac{x+\abs{x}}{2}$
     and $g\brak{x}=
     \begin{cases}
     x, & x\lt0 \\
     x^{2}, & x\geq0
     \end{cases}$.
     Then the area bounded by the curve $y=\brak{f\circ g}\brak{x}$ and the lines $y=0$, $2y-x=15$ is equal to:
    
        \begin{multicols}{4}
        \begin{enumerate}
        \item $\frac{225}{4}$
        \item $\frac{425}{4}$
        \item $\frac{325}{4}$
        \item $\frac{525}{4}$
        \end{enumerate}
        \end{multicols}

    \item Number of $4$-digit numbers that are less than or equal to $2800$ and either divisible by $3$ or by $11$, is equal to:

        \begin{multicols}{4}
        \begin{enumerate}
        \item $780$
        \item $781$
        \item $782$
        \item $783$
        \end{enumerate}
        \end{multicols}
    
\end{enumerate}
        
\end{document}
