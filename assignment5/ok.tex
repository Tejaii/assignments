\documentclass[journal]{IEEEtran}
\usepackage[a5paper, margin=10mm, onecolumn]{geometry}
\usepackage{amsmath, amssymb, amsfonts}
\usepackage{graphicx}
\usepackage{xcolor}
\usepackage{enumitem}
\usepackage{multicol}
\usepackage{mathtools}

% Define the custom commands
\newcommand{\brak}[1]{\left( #1 \right)}
\newcommand{\sbrak}[1]{\left[ #1 \right]}
\newcommand{\abs}[1]{\left| #1 \right|}
\newcommand{\lt}{\lt}
\newcommand{\gt}{\gt}

\begin{document}
\bibliographystyle{IEEEtran}
\title{ASSIGNMENT 5}
\author{EE24BTECH11034 - K Teja Vardhan}
{\let\newpage\relax\maketitle}

\section{JEE PYQ JAN 9, shift 2}
\begin{enumerate}

    \item If $ A = \sbrak{ x \in \mathbb{R} : \abs{x} \lt 2 } $ and $ B = \sbrak{ x \in \mathbb{R} : \abs{x - 2} \geq 3 } $, then:
        
        \begin{enumerate}
            \item $ A - B = \sbrak{ -1, 2 } $  
            \item $ B - A = \mathbb{R} - \sbrak{ -2, 5 } $  
            \item $ A \cup B = \mathbb{R} - \sbrak{ 2, 5 } $  
            \item $ A \cap B = \sbrak{ -2, -1 } $
        \end{enumerate}
      
    \item If 10 different balls have to be placed in 4 distinct boxes at random, then the probability that two of these boxes contain exactly 2 and 3 balls is:
        \begin{multicols}{4}
        \begin{enumerate}
            \item $ \frac{965}{2^{10}} $  
            \item $ \frac{945}{2^{10}} $  
            \item $ \frac{945}{2^{11}} $  
            \item $ \frac{965}{2^{11}} $
        \end{enumerate}
        \end{multicols}

    \item If $ x = 2 \sin \theta - \sin 2 \theta $ and $ y = 2 \cos \theta - \cos 2 \theta $, $ \theta \in \sbrak{ 0, 2\pi } $, then $ \frac{d^2y}{dx^2} $ at $ \theta = \pi $ is:
        \begin{multicols}{4}
        \begin{enumerate}
            \item $ -\frac{3}{8} $  
            \item $ \frac{3}{4} $  
            \item $ \frac{3}{2} $  
            \item $ -\frac{3}{4} $
        \end{enumerate}
        \end{multicols}

    \item Let $ f $ and $ g $ be differentiable functions on $ \mathbb{R} $, such that $ f \circ g $ is the identity function. If for some $ a, b \in \mathbb{R} $, $ g \prime \brak{a} = 5 $ and $ g(a) = b $, then $ f \prime \brak{b} $ is equal to:
        \begin{multicols}{4}
        \begin{enumerate}
            \item $ \frac{2}{5} $  
            \item $ 5 $  
            \item $ 1 $  
            \item $ \frac{1}{5} $
        \end{enumerate}
        \end{multicols}

    \item In the expansion of 
    $ \brak{ \frac{x}{\cos \theta} + \frac{1}{x \sin \theta} }^{16} $
    if $ I_1 $ is the least value of the term independent of $ x $ when $ \theta \in \sbrak{ \frac{\pi}{8}, \frac{\pi}{4} } $ and $ I_2 $ is the least value of the term independent of $ x $ when $ \theta \in \sbrak{ \frac{\pi}{16}, \frac{\pi}{8} } $, then the ratio $ I_2 : I_1 $ is equal to:
        \begin{multicols}{4}
        \begin{enumerate}
            \item $ 16:1 $  
            \item $ 8:1 $  
            \item $ 1:8 $  
            \item $ 1:16 $
        \end{enumerate}
        \end{multicols}

    \item Let $ a, b \in \mathbb{R}, a \neq 0 $, such that the equation
    $ a x^2 - 2 b x + 5 = 0 $
    has a repeated root $ \alpha $, which is also a root of the equation $ x^2 - 2 b x - 10 = 0 $. If $ \beta $ is the root of this equation, then $ a^2 + b^2 $ is equal to:
    \begin{enumerate}
        \item $ 24 $
        \item $ 25 $
        \item $ 26 $
        \item $ 28 $
    \end{enumerate}

    \item Let a function $ f: \sbrak{ 0, 5 } \rightarrow \mathbb{R} $ be continuous, $ f/brak{1} = 3 $ and $ F $ be defined as:
    $ F\brak{x} = \int_{1}^{x} t^{2} g/brak{t} dt $
    where
    $ g/brak{t} = \int_{1}^{t} f/brak{u} du $.
    Then for the function $ F $, the point $ x = 1 $ is:
    \begin{enumerate}
        \item a point of inflection.
        \item a point of local maxima.
        \item a point of local minima.
        \item not a critical point.
    \end{enumerate}

    \item Let $ [t] $ denote the greatest integer $ \leq t $ and
    $ \lim_{x \rightarrow 0} x \left[ \frac{4}{x} \right] = A $.
    Then the function, $ f\brak{x} = [x^{2}] \sin \pi x $ is discontinuous when $ x $ is equal to:
    \begin{enumerate}
        \item $ \sqrt{A+1} $
        \item $ \sqrt{A} $
        \item $ \sqrt{A+5} $
        \item $ \sqrt{A+21} $
    \end{enumerate}

    \item Let $ a - 2 b + c = 1 $.
    If $ f\brak{x} = 
    \begin{vmatrix}
    x + a & x + 2 & x + 1 \\
    x + b & x + 3 & x + 2 \\
    x + c & x + 4 & x + 3
    \end{vmatrix} $,
    then
    \begin{enumerate}
        \item $ f\brak{-50} = 501 $
        \item $ f\brak{-50} = -1 $
        \item $ f\brak{50} = 1 $
        \item $ f\brak{-50} = -501 $
    \end{enumerate}

    \item Given:
    $ f\brak{x} = 
    \begin{cases}
    x, & 0 \leq x \lt \frac{1}{2} \\
    \frac{1}{2}, & x = \frac{1}{2} \\
    1 - x, & \frac{1}{2} \lt x \leq 1
    \end{cases} $
    and $ g\brak{x} = \brak{ x - \frac{1}{2} }^{2}, x \in \mathbb{R} $.
    Then the area (in sq. units) of the region bounded by the curves $ y = f\brak{x} $ and $ y = g\brak{x} $ between the lines $ 2x = 1 $ to $ 2x = \sqrt{3} $ is:
    \begin{enumerate}
        \item $ \frac{\sqrt{3}}{4} - \frac{1}{3} $
        \item $ \frac{1}{3} + \frac{\sqrt{3}}{4} $
        \item $ \frac{1}{2} + \frac{\sqrt{3}}{4} $
        \item $ \frac{1}{2} - \frac{\sqrt{3}}{4} $
    \end{enumerate}

    \item The length of the minor axis (along y-axis) of an ellipse of the standard form is $ \frac{1}{\sqrt{3}} $. If this ellipse touches the line $ x + 6y = 8 $, then its eccentricity is:
    \begin{enumerate}
        \item $ \frac{1}{2} \left( \frac{\sqrt{5}}{3} \right) $
        \item $ \frac{1}{2} \sqrt{\frac{11}{3}} $
        \item $ \sqrt{\frac{5}{6}} $
        \item $ \frac{1}{3} \sqrt{\frac{11}{3}} $
    \end{enumerate}

    \item If $ z $ is a complex number satisfying $ \abs{\text{Re}(z)} + \abs{\text{Im}(z)} = 4 $, then $ \abs{z} $ cannot be:
    \begin{enumerate}
        \item $ \sqrt{7} $
        \item $ \sqrt{\frac{17}{2}} $
        \item $ \sqrt{10} $
        \item $ \sqrt{8} $
    \end{enumerate}

    \item If        
    $ x = \sum_{n=0}^{\infty} (-1)^{n} \tan^{2n} \theta $  
    and
    $ y = \sum_{n=0}^{\infty} \cos^{2n} \theta $
    where $ 0 \lt \theta \lt \frac{\pi}{4} $, then:
    \begin{enumerate}
        \item $ y \brak{ 1 + x } = 1 $
        \item $ x \brak{ 1 - y } = 1 $
        \item $ y \brak{ 1 - x } = 1 $
        \item $ x \brak{ 1 + y } = 1 $
        
    \end{enumerate}

\end{enumerate}

\end{document}
