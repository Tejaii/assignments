\let\negmedspace\undefined
\let\negthickspace\undefined
\documentclass[journal]{IEEEtran}
\usepackage[a5paper, margin=10mm, onecolumn]{geometry}
%\usepackage{lmodern} % Ensure lmodern is loaded for pdflatex
\usepackage{tfrupee} % Include tfrupee package

\setlength{\headheight}{1cm} % Set the height of the header box
\setlength{\headsep}{0mm}     % Set the distance between the header box and the top of the text

\usepackage{gvv-book}
\usepackage{gvv}
\usepackage{cite}
\usepackage{amsmath,amssymb,amsfonts,amsthm}
\usepackage{algorithmic}
\usepackage{graphicx}
\usepackage{textcomp}
\usepackage{xcolor}
\usepackage{txfonts}
\usepackage{listings}
\usepackage{enumitem}
\usepackage{mathtools}
\usepackage{gensymb}
\usepackage{comment}
\usepackage[breaklinks=true]{hyperref}
\usepackage{tkz-euclide} 
\usepackage{listings}
% \usepackage{gvv}                                        
\def\inputGnumericTable{}                                 
\usepackage[latin1]{inputenc}                                
\usepackage{color}                                            
\usepackage{array}                                            
\usepackage{longtable}                                       
\usepackage{calc}                                             
\usepackage{multirow}                                         
\usepackage{hhline}                                           
\usepackage{ifthen}                                           
\usepackage{lscape}

\renewcommand{\thefigure}{\theenumi}
\renewcommand{\thetable}{\theenumi}
\setlength{\intextsep}{10pt} % Space between text and floats

\numberwithin{equation}{enumi}
\numberwithin{figure}{enumi}
\renewcommand{\thetable}{\theenumi}
\newcommand{\lt}{<}
\newcommand{\gt}{>}

\begin{document}
\bibliographystyle{IEEEtran}
\title{2015-ST-14-26}
\author{EE24BTECH11034 - K Teja Vardhan}
{\let\newpage\relax\maketitle}


\begin{enumerate}



\item Let $\mathbf{u_1}, \mathbf{u_2}, \mathbf{u_3}, \mathbf{v_1}, \mathbf{v_2}, \mathbf{v_3}$ be vectors in $\mathbb{R}^4$. Let $U$ be the span of $\{\mathbf{u_1}, \mathbf{u_2}, \mathbf{u_3}\}$ and let $V$ be the span of $\{\mathbf{v_1}, \mathbf{v_2}, \mathbf{v_3}\}$.

Consider the following statements:

\begin{enumerate}[label=\brak{\Roman*}]
\item If the dimension of $U \cap V$ is $2$ and the dimension of $U$ is $3$, then $\{\mathbf{v_1}, \mathbf{v_2}, \mathbf{v_3}\}$ is linearly dependent.

\item If $U + V = \{\mathbf{u} + \mathbf{v} : \mathbf{u} \in U, \mathbf{v} \in V\} = \mathbb{R}^4$, then either $\{\mathbf{u_1}, \mathbf{u_2}, \mathbf{u_3}\}$ is linearly independent or $\{\mathbf{v_1}, \mathbf{v_2}, \mathbf{v_3}\}$ is linearly independent.
\end{enumerate}

Which of the above statements is/are true?

\begin{enumerate}[label=\brak{\Alph*}]
\item Only \brak{I}

\item Only \brak{II}

\item Both \brak{I} and \brak{II}

\item Neither \brak{I} nor \brak{II}
\end{enumerate}


\item Consider $\mathbb{R}^2$ with standard inner product. If $\mathbf{u} = \begin{bmatrix} a \\ b \end{bmatrix}$ is the vector in $\mathbb{R}^2$ such that the inner product of $\mathbf{u}$ with $\begin{bmatrix} 1 \\ 2 \end{bmatrix}$ is $2$ and with $\begin{bmatrix} 4 \\ -2 \end{bmatrix}$ is $-1$, then which one of the following statements is true?

\begin{enumerate}[label=\brak{\Alph*}]
\item Inner product of $\mathbf{u}$ with $\begin{bmatrix} 1 \\ -1 \end{bmatrix}$ is $\frac{1}{2}$

\item Inner product $\mathbf{u}$ with $\begin{bmatrix} -1 \\ 1 \end{bmatrix}$ is $\frac{3}{5}$ 

\item Inner product of $\mathbf{u}$ with $\begin{bmatrix} 2 \\ -\frac{1}{2} \end{bmatrix}$ is $-\frac{6}{5}$

\item Inner product of $\mathbf{u}$ with $\begin{bmatrix} -\frac{1}{2} \\ 1 \end{bmatrix}$ is $\frac{4}{5}$
\end{enumerate}

\item Let $A = \begin{bmatrix} a_1 & a_2 & a_3 \\ b_1 & b_2 & b_3 \end{bmatrix}$ be a $2 \times 3$ real matrix, where $\brak{a_1, a_2, a_3} \neq \brak{0,0,0}$ and $\brak{b_1, b_2, b_3} \neq \brak{0,0,0}$. Assume that the rank of $A$ is 1. Define the subspaces 

$$W = \left\{ \begin{bmatrix} x_1 \\ x_2 \\ x_3 \end{bmatrix} \in \mathbb{R}^3 : Ax = 0 \right\},$$

$$W_1 = \left\{ \begin{bmatrix} x_1 \\ x_2 \\ x_3 \end{bmatrix} \in \mathbb{R}^3 : a_1x_1 + a_2x_2 + a_3x_3 = 0 \right\},$$

and

$$W_2 = \left\{ \begin{bmatrix} x_1 \\ x_2 \\ x_3 \end{bmatrix} \in \mathbb{R}^3 : b_1x_1 + b_2x_2 + b_3x_3 = 0 \right\}.$$

Consider the following statements:

$\brak{I} W = W_1 \cap W_2$

$\brak{II} W_1 = W_2$

Which of the above statements is/are true?

\begin{enumerate}[label=\brak{\Alph*}]
\item Only \brak{I}

\item Only \brak{II}

\item Both \brak{I} and \brak{II}

\item Neither \brak{I} nor \brak{II}
\end{enumerate}

\item Let $X$ be a random variable taking only two values, $1$ and $2$. Let $M_X\brak{\cdot}$ be the moment generating function of $X$. If the expectation of $X$ is $\frac{10}{7}$, then the fourth derivative of $M_X\brak{\cdot}$ evaluated at $0$ equals

\begin{enumerate}
    \item $\frac{52}{7}$
    \item $\frac{67}{7}$
    \item $\frac{48}{7}$
    \item $\frac{60}{7}$
\end{enumerate}

\item Two fair dice, one having red and another having blue colour, are tossed independently once. Let $A$ be the event that the die having red colour will show $5$ or $6$. Let $B$ be the event that the sum of the outcomes will be $7$ and let $C$ be the event that the sum of the outcomes will be $8$. Then which one of the following statements is true?

\begin{enumerate}
    \item $A$ and $B$ are independent as well as $A$ and $C$ are independent
    \item $A$ and $B$ are independent, but $A$ and $C$ are not independent
    \item $A$ and $C$ are independent, but $A$ and $B$ are not independent
    \item Neither $A$ and $B$ are independent, nor $A$ and $C$ are independent
\end{enumerate}

\item Let $X$ be a random variable with probability density function

\[
f\brak{x} = \begin{cases}
\frac{\Gamma\brak{\alpha+\beta}}{\Gamma\brak{\alpha}\Gamma\brak{\beta}}x^{\alpha-1}\brak{1-x}^{\beta-1} & \text{if } 0 \lt x \lt 1 \\
0 & \text{otherwise,}
\end{cases} 
\]

where $\alpha \gt 0, \beta \gt 0$. If $E\brak{X} = \frac{1}{3}$ and $E\brak{X^2} = \frac{1}{6}$, then $\alpha + 3\beta$ equals

\begin{enumerate}
    \item $7$
    \item $5$
    \item $4$
    \item $8$
\end{enumerate}

\item Let $X$ and $Y$ be two random variables with cumulative distribution functions $F_X\brak{\cdot}$ and $F_Y\brak{\cdot}$, respectively. Then which one of the following statements is NOT true?

\begin{enumerate}
    \item There exist $X$ and $Y$ such that $F_X\brak{u} = F_Y\brak{u}$ for all $u \in \mathbb{R}$, and $P\brak{X \neq Y} \gt 0$.
    \item There exist $X$ and $Y$ such that $F_X\brak{u} = F_Y\brak{u}$ for all $u \in \mathbb{R}$, and $P\brak{X = Y} = 0$.
    \item If $X$ and $Y$ are independent, then $X^2$ and $Y^2$ are also independent.
    \item If $X^2$ and $Y^2$ are independent, then $X$ and $Y$ are also independent.
\end{enumerate}

\item Let $\{F_n\}_{n\geq1}$ be a sequence of cumulative distribution functions given by 

\[
F_n\brak{x} = \begin{cases}
0 & \text{if } x \lt -n \\
\frac{x+n}{2n} & \text{if } -n \leq x \lt n \\
1 & \text{if } x \geq n.
\end{cases}
\]

Which one of the following statements is true?

\begin{enumerate}
    \item $F_n\brak{x}$ converges for all $x \in \mathbb{R}$ and the limiting function is a cumulative distribution function.
    \item $F_n\brak{x}$ converges for all $x \in \mathbb{R}$, but the limiting function is not a cumulative distribution function.
    \item $F_n\brak{x}$ does not converge for any $x \in \mathbb{R}$.
    \item There exist $x, y \in \mathbb{R}$ such that $F_n\brak{x}$ converges but $F_n\brak{y}$ does not converge.
\end{enumerate}

\item Let $\{W\brak{t}\}_{t\geq 0}$ be a standard Brownian motion. Which one of the following statements is NOT true?

\begin{enumerate}
    \item $E\sbrak{W\brak{7}} = 0$
    \item $E\sbrak{W\brak{5}W\brak{9}} = 7$
    \item $2W\brak{1}$ is normally distributed with mean 0 and variance 4
    \item $E\sbrak{W\brak{5} | W\brak{3} = 3} = 3$
\end{enumerate}

\item Let $X_1, X_2, X_3$ be three independent and identically distributed binomial random variables with number of trials $n = 100$ and success probability $p \brak{0 \lt p \lt 1}$, which is an unknown parameter. Let $T_1 = \brak{X_1 + X_2, X_3}$ and $T_2 = X_1 + X_2 + X_3$. Consider the following statements:

$\brak{I}$ The distribution of $T_2$ given $T_1 = t_1$ is independent of $p$.

$\brak{II}$ The distribution of $T_1$ given $T_2 = t_2$ is independent of $p$.

Which of the above statements is/are true?

\begin{enumerate}
    \item Only $\brak{I}$
    \item Only $\brak{II}$
    \item Both $\brak{I}$ and $\brak{II}$
    \item Neither $\brak{I}$ nor $\brak{II}$
\end{enumerate}

\item Let $X_1, X_2, ..., X_n$ be a random sample of size $n \brak{\geq 2}$ from a population having probability density function

\[
f\brak{x; \theta} = \begin{cases}
\theta\brak{2x}^{\theta-1} & \text{if } 0 \lt x \leq \frac{1}{2} \\
\theta\brak{2-2x}^{\theta-1} & \text{if } \frac{1}{2} \lt x \leq 1 \\
0 & \text{otherwise,}
\end{cases}
\]

where $\theta \gt 0$ is an unknown parameter. Then which one of the following is a maximum likelihood estimator of $\theta$?

\begin{enumerate}
    \item $\sbrak{\frac{1}{n} \sum_{\{i: X_i \leq \frac{1}{2}\}} \log_e 2X_i + \sum_{\{i: X_i \gt \frac{1}{2}\}} \log_e \brak{2-2X_i}}^{-1}$
    \item $-n \sbrak{\sum_{\{i: X_i \leq \frac{1}{2}\}} \log_e 2X_i + \sum_{\{i: X_i \gt \frac{1}{2}\}} \log_e \brak{2-2X_i}}^{-1}$
    \item $n \sbrak{\sum_{1 \leq i \leq n} \log_e 2X_i + \sum_{1 \leq i \leq n} \log_e \brak{2-2X_i}}^{-1}$
    \item $-n \sbrak{\sum_{1 \leq i \leq n} \log_e 2X_i + \sum_{1 \leq i \leq n} \log_e \brak{2-2X_i}}^{-1}$
\end{enumerate}

\item In a testing of hypothesis problem, which one of the following statements is true?

\begin{enumerate}
    \item The probability of the Type-I error cannot be higher than the probability of the Type-II error.
    \item Type-II error occurs if the test accepts the null hypothesis when the null hypothesis is actually false.
    \item Type-I error occurs if the test rejects the null hypothesis when the null hypothesis is actually false.
    \item The sum of the probability of the Type-I error and the probability of the Type-II error should be 1.
\end{enumerate}

\item A random sample of size $40$ is drawn from a population having four distinct categories as $i = 1, 2, 3, 4$. The data are given as

\begin{tabular}{|c|c|c|c|c|}
\hline
Category & 1 & 2 & 3 & 4 \\
\hline
Observed Frequency & 5 & 8 & 12 & 15 \\
\hline
\end{tabular}

Let $\theta_i$ be the probability that an observation comes from the $i$-th category, $i = 1, 2, 3, 4$. If the chi-square goodness-of-fit test is used to test $H_0: \theta_i = \frac{1}{4}, i = 1, 2, 3, 4$ against $H_1: \theta_i \neq \frac{1}{4}$ for some $i = 1, 2, 3, 4$, then which one of the following statements is true?

\begin{enumerate}
    \item Under $H_0$, the test statistic follows central chi-square distribution with $3$ degrees of freedom and the observed value of the test statistic is $5.8$.
    \item Under $H_0$, the test statistic follows central chi-square distribution with $3$ degrees of freedom and the observed value of the test statistic is $1.4$.
    \item Under $H_0$, the test statistic follows central chi-square distribution with $4$ degrees of freedom and the observed value of the test statistic is $5.8$.
    \item Under $H_0$, the test statistic follows central chi-square distribution with $4$ degrees of freedom and the observed value of the test statistic is $1.4$.
\end{enumerate}
\end{enumerate}
\end{document}