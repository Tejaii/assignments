\documentclass[journal]{IEEEtran}
\usepackage[a5paper, margin=10mm, onecolumn]{geometry}
\usepackage{amsmath, amssymb, amsfonts}
\usepackage{graphicx}
\usepackage{xcolor}
\usepackage{enumitem}
\usepackage{multicol}
\usepackage{mathtools}

% Define the custom commands
\newcommand{\brak}[1]{\left( #1 \right)}
\newcommand{\sbrak}[1]{\left[ #1 \right]}
\newcommand{\abs}[1]{\left| #1 \right|}
\newcommand{\lt}{<}
\newcommand{\gt}{>}

\begin{document}
\bibliographystyle{IEEEtran}
\title{ASSIGNMENT 6}
\author{EE24BTECH11034 - K Teja Vardhan}
{\let\newpage\relax\maketitle}

\section{JEE PYQ JAN 20, shift 1}
\begin{enumerate}
        
    \item Let $y = mx + c$, $m \gt 0$ be the focal chord of $y^2 = -64x$ which is tangent to $\brak{x + 10}^2 + y^2 = 4$. Then the value of $4\sqrt{2} \brak{m + c}$ is equal to
    
        \begin{multicols}{4}
        \begin{enumerate}
        \item $16$
        \item $24$
        \item $34$
        \item $40$
        \end{enumerate}    
        \end{multicols}
        
    \item A continuous differentiable function $f\brak{x}$ is increasing in $\brak{-\infty, \frac{3}{2}}$ and decreasing in $\brak{\frac{3}{2}, \infty}$. Then $x = \frac{3}{2}$ is
        
        \begin{enumerate}
        \item a point of local maxima
        \item a point of local minima
        \item a point of inflection
        \item None of these
        \end{enumerate}

    \item If $z$ and $w$ are complex numbers such that $\abs{z \omega} = 1$, $\arg\brak{z} - \arg\brak{w} = \frac{3\pi}{2}$, then find $\arg\brak{\frac{1 - 2z\omega}{1 + 3z\omega}}$
    
        \begin{multicols}{4}
        \begin{enumerate}
        \item $\frac{\pi}{4}$
        \item $-\frac{\pi}{4}$
        \item $\frac{3\pi}{4}$
        \item $-\frac{3\pi}{4}$
        \end{enumerate}
        \end{multicols}
        
    \item If an invertible function $f\brak{x}$ is defined as $f\brak{x} = 3x - 2$, and $g\brak{x}$ is also an invertible function such that $f^{-1}\brak{g^{-1}\brak{x}} = x - 2$, then $g\brak{x}$ is
    
        \begin{multicols}{4}
        \begin{enumerate}
        \item $\frac{x - 8}{3}$
        \item $\frac{x + 8}{3}$
        \item $\frac{x - 3}{8}$
        \item $\frac{x + 3}{8}$
        \end{enumerate}
        \end{multicols}
        
    \item The probability of selecting integers $a \in \sbrak{-5, 30}$, such that $x^2 + 2\brak{a + 4}x - 5a + 64 \gt 0$ for all $x \in \mathbb{R}$ is:
    
        \begin{multicols}{4}
        \begin{enumerate}
        \item $\frac{1}{2}$
        \item $\frac{1}{3}$
        \item $\frac{1}{4}$
        \item $\frac{1}{5}$
        \end{enumerate}
        \end{multicols}
        
    \item If $\int_0^a e^{x - \sbrak{x}} dx = 10e - 9$, then the value of $a$ is     
        \begin{multicols}{4}
        \begin{enumerate}
        \item $9 + \ln 2$
        \item $10 + \ln 2$
        \item $10$
        \item $9$
        \end{enumerate}
        \end{multicols}

    \item If the shortest distance between the lines

        $\mathbf{r_1} = \alpha \hat{i} + 2 \hat{j} + 2 \hat{k} + \lambda \brak{\hat{i} - 2 \hat{j} + 2 \hat{k}}$
        
        , $\lambda \in \mathbb{R}$, $\alpha \gt 0$ and
        
        $\mathbf{r_2} = -4 \hat{i} - \hat{k} + \mu \brak{3 \hat{i} - 2 \hat{j} - 2 \hat{k}}$
        
        , $\mu \in \mathbb{R}$ is 9, then the value of $\alpha$ is:
        
        \begin{multicols}{4}
        \begin{enumerate}
        \item $2$
        \item $4$
        \item $6$
        \item $\sqrt{6}$
        \end{enumerate}
        \end{multicols}
        
        
    \item Let $a_{ij} = \begin{cases}
        1, & i = j \\
        -x, & \abs{i - j} = 1 \\
        2x + 1, & \text{otherwise}
        \end{cases}$
        
        , $A = \sbrak{a_{ij}}_{3 \times 3} = \det\brak{A}$. Then find the sum of local maximum and minimum values of $f\brak{x}$.
        
        \begin{multicols}{4}
        \begin{enumerate}
        \item $\frac{20}{27}$
        \item $-\frac{20}{27}$
        \item $\frac{88}{27}$
        \item $-\frac{88}{27}$
        \end{enumerate}
        \end{multicols}

    \item Find the coefficient of $a^3b^4c^5$ in $\brak{ab + bc + ca}^6$.
        
        \begin{multicols}{4}
        \begin{enumerate}
        \item $60$
        \item $45$
        \item $40$
        \item $90$
        \end{enumerate}
        \end{multicols}
        
    
     \item $x \brak{\frac{dy}{dx}} \tan\brak{\frac{y}{x}} = y \tan\brak{\frac{y}{x}} + x$, $y\brak{\frac{1}{2}} = \frac{\pi}{6}$. The area bounded by $x = 0$, $x = \frac{1}{\sqrt{2}}$, and $y = y\brak{x}$ is:
    
        \begin{multicols}{4}
        \begin{enumerate}
        \item $\frac{\pi - 1}{8}$
        \item $\frac{\pi - 2}{16}$
        \item $\frac{\pi - 3}{32}$
        \item $\frac{\pi - 4}{64}$
        \end{enumerate}
        \end{multicols}

    \end{enumerate}
\end{document}
