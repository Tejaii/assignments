\documentclass[journal]{IEEEtran}
\usepackage[a5paper, margin=10mm, onecolumn]{geometry}
\usepackage{amsmath, amssymb, amsfonts}
\usepackage{graphicx}
\usepackage{xcolor}
\usepackage{enumitem}
\usepackage{multicol}
\usepackage{mathtools}

% Define the custom commands
\newcommand{\brak}[1]{\left( #1 \right)}
\newcommand{\sbrak}[1]{\left[ #1 \right]}
\newcommand{\abs}[1]{\left| #1 \right|}
\newcommand{\lt}{<}
\newcommand{\gt}{>}

\begin{document}
\bibliographystyle{IEEEtran}
\title{ASSIGNMENT 8}
\author{EE24BTECH11034 - K Teja Vardhan}
{\let\newpage\relax\maketitle}

\section{JEE PYQ 2022 June 26, shift 1}
\begin{enumerate}

   \item If $\vec{a}\cdot\vec{b}=1$, $\vec{b}\cdot\vec{c}=2$ and $\vec{c}\cdot\vec{a}=3$, then the value of $\sbrak{\vec{a}\times\brak{\vec{b}\times\vec{c}}, \vec{b}\times\brak{\vec{c}\times\vec{a}}, \vec{c}\times\brak{\vec{b}\times\vec{a}}}$ is:

        \begin{multicols}{4}
        \begin{enumerate}
        \item $0$
        \item $-6\vec{a}\cdot\brak{\vec{b}\times\vec{c}}$
        \item $12\vec{c}\cdot\brak{\vec{a}\times\vec{b}}$
        \item $-12\vec{b}\cdot\brak{\vec{c}\times\vec{a}}$
        \end{enumerate}
        \end{multicols} 

    \item Let a biased coin be tossed $5$ times. If the probability of getting $4$ heads is equal to the probability of getting $5$ heads, then the probability of getting at most two heads is:
        
        \begin{multicols}{4}
        \begin{enumerate}
        \item $\frac{275}{6^{5}}$
        \item $\frac{36}{5^{4}}$
        \item $\frac{181}{5^{5}}$
        \item $\frac{46}{6^{4}}$
        \end{enumerate}
        \end{multicols}
        
    
    \item The mean of the numbers $a$, $b$, $8$, $5$, $10$ is $6$ and their variance is $6.8$. If $M$ is the mean deviation of the numbers about the mean, then $25M$ is equal to:

        \begin{multicols}{4}
        \begin{enumerate}
        \item $60$
        \item $55$
        \item $50$
        \item $45$
        \end{enumerate}
        \end{multicols}

    \item Let $f\brak{x}=2\cos^{-1}x+4\cot^{-1}x-3x^{2}-2x+10$, $x\in\sbrak{-1,1}$. If $\sbrak{a,b}$ is the range of the function, then $4a-b$ is equal to:

        \begin{multicols}{4}
        \begin{enumerate}
        \item $11$
        \item $11-\pi$
        \item $11+\pi$
        \item $15-\pi$
        \end{enumerate}
        \end{multicols}

    \item Let $\land$, $\lor$, $\epsilon$ $\sbrak{\land,\lor}$ be such that $p\lor q\Rightarrow\sbrak{\brak{p\land q}\lor r}$ is a tautology. Then $\brak{p\lor q}\land r$ is logically equivalent to:

        \begin{multicols}{4}
        \begin{enumerate}
        \item $\sbrak{\brak{p\land r}\lor q}$
        \item $\sbrak{\brak{p\land r}\land q}$
        \item $\sbrak{\brak{p\land r}\land q}$
        \item $\sbrak{\brak{p\lor r}\land q}$
        \end{enumerate}
        \end{multicols}

    \item The sum of the cubes of all the roots of the equation $x^{4}-3x^{3}-2x^{2}+3x+1=10$ is:

        \begin{multicols}{4}
        \begin{enumerate}
        \item $10$
        \item $27$
        \item $36$
        \item $45$
        \end{enumerate}
        \end{multicols}

    \item There are ten boys $B_1, B_2, \ldots, B_{10}$ and five girls $G_1, G_2, \ldots, G_5$ in a class. Then the number of ways of forming a group consisting of three boys and three girls, if both $B_1$ and $B_2$ together should not be the members of a group, is:

        \begin{multicols}{4}
        \begin{enumerate}
        \item $1120$
        \item $960$
        \item $1080$
        \item $1200$
        \end{enumerate}
        \end{multicols}

    \item Let the common tangents to the curves $4\brak{x^{2}+y^{2}}=9$ and $y^{2}=4x$ intersect at the point $Q$. Let an ellipse, centered at the origin $O$, has lengths of semi-minor and semi-major axes equal to $OQ$ and $6$, respectively. If $e$ and $l$ respectively denote the eccentricity and the length of the latusrectum of this ellipse, then $\frac{l}{e^{2}}$ is equal to:

        \begin{multicols}{4}
        \begin{enumerate}
        \item $16$
        \item $32$
        \item $25$
        \item $4$
        \end{enumerate}
        \end{multicols}

    \item Let $f\brak{x}=\max\brak{\abs{x+1},\abs{x+2},\ldots,\abs{x+5}}$. Then $\int_{-6}^{0}f\brak{x}dx$ is equal to:

        \begin{multicols}{4}
        \begin{enumerate}
        \item $11$
        \item $21$
        \item $13$
        \item $23$
        \end{enumerate}
        \end{multicols}

    \item Let the solution curve $y=y\brak{x}$ of the differential equation $\brak{4+x^{2}}dy-2x\brak{x^{2}+3y+4}dx=0$ pass through the origin. Then $y\brak{2}$ is equal to:

        \begin{multicols}{4}
        \begin{enumerate}
        \item $11$
        \item $12$
        \item $13$
        \item $14$
        \end{enumerate}
        \end{multicols}

    \item If $\sin^{2}\brak{10^{\circ}}\sin\brak{20^{\circ}}\sin\brak{40^{\circ}}\sin\brak{50^{\circ}}\sin\brak{70^{\circ}}=\alpha-\frac{1}{16}\sin\brak{10^{\circ}}$, then $16+\alpha^{-1}$ is equal to:

        \begin{multicols}{4}
        \begin{enumerate}
        \item $20$
        \item $40$
        \item $80$
        \item $160$
        \end{enumerate}
        \end{multicols}

    \item Let $A=\sbrak{n\in\mathbb{N}:\text{H.C.F.}\brak{n,45}=1}$ and $B=\sbrak{2k:k\in\sbrak{1,2,\ldots,100}}$. Then the sum of all the elements of $A\cap B$ is:

        \begin{multicols}{4}
        \begin{enumerate}
        \item $1000$
        \item $1020$
        \item $1040$
        \item $1060$
        \end{enumerate}
        \end{multicols}

    \item The value of the integral $\frac{48}{\pi^{4}}\int_{0}^{\pi}\brak{\frac{3\pi x^{2}}{2}-x^{3}}\frac{\sin x}{1+\cos^{2}x}dx$ is equal to:
    
        \begin{multicols}{4}
        \begin{enumerate}
        \item $9$
        \item $12$
        \item $15$
        \item $18$
        \end{enumerate}
        \end{multicols}
        
    \item Let $A=\sum_{i=1}^{10}\sum_{j=1}^{10}\min\brak{i,j}$ and $B=\sum_{i=1}^{10}\sum_{j=1}^{10}\max\brak{i,j}$. Then $A+B$ is equal to:
    
        \begin{multicols}{4}
        \begin{enumerate}
        \item $1000$
        \item $1100$
        \item $1200$
        \item $1300$
        \end{enumerate}
        \end{multicols}
        
    \item Let $S=\left(0,2\pi\right)-\sbrak{\frac{\pi}{2},\frac{3\pi}{4},\frac{3\pi}{2},\frac{7\pi}{4}}$. Let $y=y\brak{x}$, $x\in S$, be the solution curve of the differential equation $\frac{dy}{dx}=\frac{1}{1+\sin 2x}$, $y\brak{\frac{\pi}{4}}=\frac{1}{2}$. If the sum of abscissas of all the points of intersection of the curve $y=y\brak{x}$ with the curve $y=\sqrt{2}\sin x$ is $\frac{k\pi}{12}$, then $k$ is equal to:
    
        \begin{multicols}{4}
        \begin{enumerate}
        \item $11$
        \item $13$
        \item $15$
        \item $17$
        \end{enumerate}
        \end{multicols}
        
    

\end{enumerate}   
\end{document}
