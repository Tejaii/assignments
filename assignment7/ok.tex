\documentclass[journal]{IEEEtran}
\usepackage[a5paper, margin=10mm, onecolumn]{geometry}
\usepackage{amsmath, amssymb, amsfonts}
\usepackage{graphicx}
\usepackage{xcolor}
\usepackage{enumitem}
\usepackage{multicol}
\usepackage{mathtools}

% Define the custom commands
\newcommand{\brak}[1]{\left( #1 \right)}
\newcommand{\sbrak}[1]{\left[ #1 \right]}
\newcommand{\abs}[1]{\left| #1 \right|}
\newcommand{\lt}{<}
\newcommand{\gt}{>}


\begin{document}
\bibliographystyle{IEEEtran}
\title{ASSIGNMENT 7}
\author{EE24BTECH11034 - K Teja Vardhan}
{\let\newpage\relax\maketitle}

\section{JEE PYQ JAN 25, shift 2}
\begin{enumerate}

    \item The sum of all those terms which are rational numbers in the expansion of $\brak{2^{1/3}+3^{1/4}}^{12}$ is:

        \begin{multicols}{4}
        \begin{enumerate}
        \item $89$
        \item $27$
        \item $35$
        \item $43$
        \end{enumerate}
        \end{multicols}
        
    \item The first of the two samples in a group has $100$ items with mean $15$ and standard deviation $3$. If the whole group has $250$ items with mean $15.6$ and standard deviation $\sqrt{13.44}$, then the standard deviation of the second sample is:

        \begin{multicols}{4}
        \begin{enumerate}
        \item $8$
        \item $6$
        \item $4$
        \item $5$
        \end{enumerate}
        \end{multicols}
        
    \item If $f\brak{x}=\begin{cases}\int_{0}^{1}\brak{5+\abs{1-t}}dt,&x\gt2\\ 5x+1,&x\le2\end{cases}$, then

        \begin{enumerate}
        \item $f\brak{x}$ is not continuous at $x=2$
        \item $f\brak{x}$ is everywhere differentiable
        \item $f\brak{x}$ is continuous but not differentiable at $x=2$
        \item $f\brak{x}$ is not differentiable at $x=1$
        \end{enumerate}
    
    \item If the greatest value of the term independent of $x$ in the expansion of $\brak{x\sin\alpha+a+\frac{\cos\alpha}{x}}^{10}$ is $\frac{10!}{\brak{5!}^2}$, then the value of $a$ is equal to:
    
        \begin{multicols}{4}
        \begin{enumerate}
        \item $-1$
        \item $1$
        \item $-2$
        \item $2$
        \end{enumerate}
        \end{multicols}

    \item Consider the statement: The match will be played only if the weather is good and the ground is not wet. Select the correct negation from the following:
    
   
        \begin{enumerate}
        \item The match will not be played and the weather is not good and the ground is wet.
        \item If the match will not be played, then either the weather is not good or the ground is wet.
        \item The match will be played and the weather is not good or the ground is wet.
        \item The match will not be played or the weather is good and the ground is not wet.
        \end{enumerate}
        

    \item The value of $\cot\frac{\pi}{24}$ is:
        
        \begin{enumerate}
        \item $\sqrt{2}+\sqrt{3}+2-\sqrt{6}$
        \item $\sqrt{2}+\sqrt{3}+2+\sqrt{6}$
        \item $\sqrt{2}-\sqrt{3}-2+\sqrt{6}$
        \item $3\sqrt{2}-\sqrt{3}-\sqrt{6}$
        \end{enumerate}
        
    \item The lowest integer which is greater than $\brak{1+\frac{1}{10^{100}}}^{10^{100}}$ is:

        \begin{multicols}{4}
        \begin{enumerate}
        \item $3$
        \item $4$
        \item $2$
        \item $1$
        \end{enumerate}
        \end{multicols}

    \item The value of the integral $\int_{-1}^{1}\log\brak{x+\sqrt{x^2+1}}dx$ is:
    
        \begin{multicols}{4}
        \begin{enumerate}
        \item $2$
        \item $0$
        \item $-1$
        \item $1$
        \end{enumerate}
        \end{multicols}

    \item Let $a$, $b$, and $c$ be distinct positive numbers. If the vectors $\hat{a}i+\hat{a}j+\hat{c}k$, $\hat{i}+\hat{k}$, and $\hat{c}i+\hat{c}j+\hat{b}k$ are co-planar, then $c$ is equal to:

        \begin{multicols}{4}
        \begin{enumerate}
        \item $\frac{2}{\frac{1}{a}+\frac{1}{b}}$
        \item $\frac{a+b}{2}$
        \item $\frac{1}{a}+\frac{1}{b}$
        \item $\sqrt{ab}$
        \end{enumerate}
        \end{multicols}

    \item If $\sbrak{x}$ be the greatest integer less than or equal to $x$, then $\sum_{n=8}^{100}\frac{\brak{-1}^n\sbrak{n}}{2}$ is equal to:

        \begin{multicols}{4}
        \begin{enumerate}
        \item $0$
        \item $4$
        \item $-2$
        \item $2$
        \end{enumerate}
        \end{multicols}

    \item The number of distinct real roots of $\begin{vmatrix}\sin x & \cos x & \cos x \\ \cos x & \sin x & \cos x \\ \cos x & \cos x & \sin x\end{vmatrix}=0$ in the interval $\frac{\pi}{4}\le x\le\frac{\pi}{4}$ is:

        \begin{multicols}{4}
        \begin{enumerate}
        \item $4$
        \item $1$
        \item $2$
        \item $3$
        \end{enumerate}
        \end{multicols}

    \item If $\abs{a}=2$, $\abs{b}=5$ and $\abs{a\times b}=8$, then $\abs{a\cdot b}$ is equal to:

        \begin{multicols}{4}
        \begin{enumerate}
        \item $6$
        \item $4$
        \item $3$
        \item $5$
        \end{enumerate}
        \end{multicols}

    \item The number of real solutions of the equation, $x^2-\abs{x}-12=0$ is:

        \begin{multicols}{4}
        \begin{enumerate}
        \item $2$
        \item $3$
        \item $1$
        \item $4$
        \end{enumerate}
        \end{multicols}

    \item Consider functions $f:A\rightarrow B$ and $g:B\rightarrow C$ $\brak{A,B,C\subseteq\mathbb{R}}$ such that $\brak{gof}^{-1}$ exists, then:

        \begin{enumerate}
        \item $f$ and $g$ both are one-one
        \item $f$ and $g$ both are onto
        \item $f$ is one-one and $g$ is onto
        \item $f$ is onto and $g$ is one-one
        \end{enumerate}

    \item If $P=\begin{bmatrix}1 & 0 \\ \frac{1}{2} & 1\end{bmatrix}$, then $P^{50}$ is:

        \begin{multicols}{4}
        \begin{enumerate}
        \item $\begin{bmatrix}1 & 0 \\ 25 & 1\end{bmatrix}$
        \item $\begin{bmatrix}1 & 50 \\ 0 & 1\end{bmatrix}$
        \item $\begin{bmatrix}1 & 25 \\ 0 & 1\end{bmatrix}$
        \item $\begin{bmatrix}1 & 0 \\ 50 & 1\end{bmatrix}$
        \end{enumerate}
        \end{multicols}
                
\end{enumerate}
\end{document}
